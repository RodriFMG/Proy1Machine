\documentclass[10pt,letterpaper]{article}
\setlength{\parindent}{0em} 
\setlength{\parskip}{0.5em}   
\textwidth 6.5in
\textheight 9.in
\oddsidemargin 0in
\headheight 0in
\usepackage{booktabs}


\usepackage{colortbl}  % Para colorear la tabla
\usepackage{xcolor}    % Para definir colores
\usepackage{array}     % Mejor alineación de columnas
\usepackage{booktabs}  % Bordes refinados
\usepackage{cellspace} % Espaciado dentro de celdas

\definecolor{headercolor}{RGB}{60, 76, 110} % Azul oscuro
\definecolor{rowcolor}{RGB}{230, 240, 250}  % Azul claro


\usepackage{fancybox}
\usepackage[utf8]{inputenc}
\usepackage{epsfig,graphicx}
\usepackage{multicol,pst-plot}
\usepackage{pstricks}
\usepackage{amsmath}
\usepackage{amsfonts}
\usepackage{amssymb}
\usepackage{eucal}
\usepackage[left=2cm,right=2cm,top=2cm,bottom=2cm]{geometry}
\usepackage{txfonts}
\usepackage[spanish]{babel}
\usepackage[colorlinks]{hyperref}
\usepackage{cancel}
\usepackage{caption}
\usepackage{float}
\usepackage{upgreek}
\usepackage{gensymb}
\usepackage{subfigure}
\usepackage{siunitx}
\usepackage{color}
\usepackage{tikz}
\usepackage{listings}
%\usepackage{minted}
\usepackage{mdframed}
\definecolor{codegreen}{rgb}{0,0.6,0}
\definecolor{codegray}{rgb}{0.5,0.5,0.5}
\definecolor{backcolour}{rgb}{0.95,0.95,0.95}




\lstset{ %
	language=python,                % choose the language of the code
	basicstyle=\footnotesize,       % the size of the fonts that are used for the code
	numbers=left,                   % where to put the line-numbers
	numberstyle=\footnotesize,      % the size of the fonts that are used for the line-numbers
	stepnumber=1,                   % the step between two line-numbers. If it is 1 each line will be numbered
	numbersep=5pt,                  % how far the line-numbers are from the code
	backgroundcolor=\color{white},  % choose the background color. You must add \usepackage{color}
	showspaces=false,               % show spaces adding particular underscores
	showstringspaces=false,         % underline spaces within strings
	showtabs=false,                 % show tabs within strings adding particular underscores
	frame=single,                   % adds a frame around the code
	tabsize=2,                      % sets default tabsize to 2 spaces
	captionpos=b,                   % sets the caption-position to bottom
	breaklines=true,                % sets automatic line breaking
	breakatwhitespace=false,        % sets if automatic breaks should only happen at whitespace
	escapeinside={\%*}{*)}          % if you want to add a comment within your code
}
\lstdefinestyle{mystyle}{
	backgroundcolor=\color{backcolour},   
	commentstyle=\color{red},
	keywordstyle=\bfseries\color{magenta},
	numberstyle=\tiny\color{codegray},
	stringstyle=\color{codegreen},
	basicstyle=\footnotesize\ttfamily,
	identifierstyle=\color{blue},
	breakatwhitespace=false,         
	breaklines=true,                 
	captionpos=b,                    
	keepspaces=true,                 
	numbers=left,                    
	numbersep=5pt,                  
	showspaces=false,                
	showstringspaces=false,
	showtabs=false,                  
	tabsize=2
}
\lstset{style=mystyle}




%CONFIGURACIÓN DE MINTED PARA CÓDIGOS

%\usemintedstyle{vs}

%DEFINICIÓN DE COMANDOS EXTRAS

\pagestyle{empty}
\DeclareMathOperator{\tr}{Tr}                      %ICONO TRAZA MECANICA CUANTICA
\DeclareMathOperator{\rsol}{R_\odot}               %ICONO RADIO SOLAR
\DeclareMathOperator{\lsol}{L_\odot}               %ICONO LUMINOSIDAD SOLAR
\DeclareMathOperator{\msol}{M_\odot}               %ICONO MASA SOLAR
\DeclareMathOperator{\probabi}{Prob}               %ICONO PROBABILIDAD
\newcommand{\units}[1]{\left[ #1 \right]}          %CORCHETES PARA UNIDADES
\newcommand{\prob}[1]{\probabi\left( #1 \right)}   %OPERADOR PROBABILIDAD
\newcommand{\abs}[1]{\left|#1\right|}              %OPERADOR VALOR ABSOLUTO
\newcommand{\bra}[1]{\langle #1 |}                 %OPERADOR BRA
\newcommand{\ket}[1]{| #1 \rangle}                 %OPERADOR KET
\newcommand{\braket}[2]{\langle #1 | #2 \rangle}   %OPERADOR BRA-KET
\newcommand{\ketbra}[2]{|#1\rangle\langle#2|}      %OPERADOR KET-BRA
\newcommand{\mean}[1]{\langle #1 \rangle}          %PROMEDIO MECANICA CUANTICA
\newcommand{\eval}[3]{\left.#1\right|_{#2}^{#3}}   %COMANDO PARA EVALUAR INTEGRALES
